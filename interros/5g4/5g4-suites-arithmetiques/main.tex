\documentclass[12pt]{article}

\usepackage{fontspec}
\usepackage{amsmath, amssymb, amsfonts}
\usepackage{graphicx}
\usepackage{hyperref}
\usepackage{color}
\usepackage{geometry}
\geometry{a4paper}

\usepackage{tikz}
\usetikzlibrary{calc, positioning}

\begin{document}
	\begin{minipage}[t]{.5\textwidth}
		{\large Prénom :\\
		Nom :}
	\end{minipage}%
	\begin{minipage}[t]{.5\textwidth}
		\begin{flushright}
			{\Large /10}
		\end{flushright}
	\end{minipage}

	\vspace{3em}
	\begin{center}
			{\Large Interrogation : les suites arithmétiques}\\
			{\large 5e Générale}\\
			22 novembre 2023
	\end{center}
	
	\vspace{3em}
	
	\textbf{Consignes :} Tu peux écrire sur cette feuille ou sur une feuille à part, n'oublie pas de bien écrire tes prénom et nom sur toutes les feuilles que tu utilises. Les machines à calculer sont autorisées. Pose des questions si tu en as besoin. Bon courage !
	
	\vspace{2em}
	
	\begin{enumerate}
		\item 
			\begin{minipage}[t]{.9\textwidth}
				Donne une des deux formules permettant de calculer un élément $u_n$ d'une suite arithmétique en connaissant la valeur initiale $u_0$ et la raison $r$.
			\end{minipage}%
			\begin{minipage}{.1\textwidth}
				\begin{flushright}
					{\large /1}
				\end{flushright}
			\end{minipage}
			\vspace{3em}
			
		\item 
			\begin{minipage}[t]{.9\textwidth}
				Soit la suite arithmétique $u$ définie par $u_0 = -3$ et $r=2$. Calcule $u_{234}$.
			\end{minipage}%
			\begin{minipage}{.1\textwidth}
				\begin{flushright}
					{\large /1}
				\end{flushright}
			\end{minipage}
			\vspace{3em}
			
		\item 
			\begin{minipage}[t]{.9\textwidth}
				Soit une suite arithmétique $u$. On sait que $u_6 = 8$ et $u_8 = 12$. Calcule $u_{11}$.
			\end{minipage}%
			\begin{minipage}{.1\textwidth}
				\begin{flushright}
					{\large /2}
				\end{flushright}
			\end{minipage}
			\vspace{3em}
			
		\item 
			\begin{minipage}[t]{.9\textwidth}
				Est-ce que les suites suivantes sont arithmétiques ? Si oui, donne la raison.
				\begin{enumerate}
					\item $u_0 = 81, \quad u_1 = 27, \quad u_2 = 9, \quad u_3 = 3, \quad u_4 = 1,\dots$
					\item $u_0 = 4, \quad u_1 = \frac{7}{2}, \quad u_2 = 3, \quad u_3 = \frac{5}{2}, \quad u_4 = 2,\dots$
				\end{enumerate}
			\end{minipage}%
			\begin{minipage}{.1\textwidth}
				\begin{flushright}
					{\large /2}
				\end{flushright}
			\end{minipage}
			\vspace{3em}
			
		\item 
			\begin{minipage}[t]{.9\textwidth}
				Soit une suite $u$. On a $u_0 = 7$, $u_1=4$, $u_2 = 6$ et $u_3 = 5$. Calcule $\sum_{n=0}^{3} u_n$.
			\end{minipage}%
			\begin{minipage}{.1\textwidth}
				\begin{flushright}
					{\large /1}
				\end{flushright}
			\end{minipage}
			\vspace{3em}
			
		\item 
			\begin{minipage}[t]{.9\textwidth}
				Soit la suite arithmétique $u$ définie par $u_0 = 5$ et $r = 3$. Calcule $S_{243}$, la somme les 243 premiers éléments de la suite $u$.
			\end{minipage}%
			\begin{minipage}{.1\textwidth}
				\begin{flushright}
					{\large /3}
				\end{flushright}
			\end{minipage}
			\vspace{3em}
	\end{enumerate}
	
\end{document}