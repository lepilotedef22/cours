\documentclass[12pt]{article}

\usepackage{fontspec}
\usepackage{amsmath, amssymb, amsfonts}
\usepackage{graphicx}
\usepackage{hyperref}
\usepackage{color}
\usepackage{geometry}
\geometry{a4paper}
\usepackage{pgfplots}

\begin{document}
	\begin{minipage}[t]{.5\textwidth}
		{\large Prénom :\\
		Nom :}
	\end{minipage}%
	\begin{minipage}[t]{.5\textwidth}
		\begin{flushright}
			{\Large /5}
		\end{flushright}
	\end{minipage}

	\vspace{3em}
	\begin{center}
			{\Large Interrogation : les sommes de suites arithmétiques}\\
			{\large 6e Technique}\\
			23 novembre 2023
	\end{center}
	
	\vspace{3em}
	
	\textbf{Consignes :} Tu peux écrire sur cette feuille ou sur une feuille à part, n'oublie pas de bien écrire tes prénom et nom sur toutes les feuilles que tu utilises. Les machines à calculer sont autorisées. Pose des questions si tu en as besoin. Bon courage !
	
	\vspace{2em}
	
	\begin{enumerate}
		\item 
			\begin{minipage}[t]{.9\textwidth}
				Donne la formule qui permet de calculer $S_N$, la somme des $N$ premiers éléments d'une suite arithmétique $u$.
			\end{minipage}%
			\begin{minipage}{.1\textwidth}
				\begin{flushright}
					{\large /1}
				\end{flushright}
			\end{minipage}
			\vspace{3em}
			
		\item 
			\begin{minipage}[t]{.9\textwidth}
				Soit la suite arithmétique $u$ définie par $u_0 = 3$ et $r = 5$. Calcule $S_{31}$, la somme des 31 premiers éléments de la suite $u$.
			\end{minipage}%
			\begin{minipage}{.1\textwidth}
				\begin{flushright}
					{\large /2}
				\end{flushright}
			\end{minipage}
			\vspace{3em}
			
		\item 
			\begin{minipage}[t]{.9\textwidth}
				Calcule la somme des nombres de 1 à 100.
			\end{minipage}%
			\begin{minipage}{.1\textwidth}
				\begin{flushright}
					{\large /2}
				\end{flushright}
			\end{minipage}
			\vspace{3em}

	\end{enumerate}
	
\end{document}