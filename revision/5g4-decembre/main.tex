\documentclass[12pt]{article}

\usepackage{fontspec}
\usepackage{amsmath, amssymb, amsfonts}
\usepackage{graphicx}
\usepackage{hyperref}
\usepackage{color}
\usepackage{geometry}
\geometry{a4paper}
\usepackage{pgfplots}

\title{Mathématiques - révisions}
\date{Décembre 2023}

\begin{document}

	\maketitle

	\begin{enumerate}
		\item Calcule les 5 premiers éléments des suites suivantes :
		\begin{enumerate}
			\item $u_n = 2n^2 - 1$
			\item $u_n = u_{n-1}^2; ~ u_0 = 2$
			\item $u_n = -4n^3 + 2n + 1$
		\end{enumerate}
	
		\item Pour chacune des suites $u$, exprime le terme général $u_n$ en fonction de $n$ :
			\begin{enumerate}
				\item $1, -2, 3, -4, 5, -6, \dots$
				\item $\frac{1}{2}, \frac{2}{3}, \frac{3}{4}, \frac{4}{5}, \dots$
				\item $1, 3, 7, 15, 31, 63, \dots$
			\end{enumerate}
	
		\item Les suites suivantes sont-elles croissantes ?
			\begin{enumerate}
				\item $u_n = 3n^2 - 6$
				\item $u_n = -6n^3$
				\item $u_n = u_{n-1} + 1; ~u_0 = 0$
				\item $u_n = 4u_{n-1}; ~u_0 = 3$
			\end{enumerate}
		
		\item Pour quelles valeurs de $u_0$ la suite récurrente $u_n = \left(u_{n-1}\right)^2$ est-elle croissante ?
		
		\item Démontre que la suite $u_n = 4n - 3$ est strictement croissante.
		
		\item Soit $u$ une suite croissante. 
			\begin{enumerate}
				\item Est-ce qu'il y a un élément maximal à cette suite ?
				\item Quel est le plus petit élément de la suite ?
			\end{enumerate}
		
		\item Soit la suite arithmétique définie par $u_0 = 4$ et $r = 5$. Calcule $u_{267}$.
		
		\item Soit $u$ une suite arithmétique. Exprime $u_{12}$ en fonction de $u_{11}$ et $u_{13}$. Combien vaut $u_{12}$ si $u_{11} = 15$ et $u_{13} = 11$ ?
		
		\item Soit $u$ une suite arithmétique. Si $u_{24} = 50$ et $u_{28} = 70$, combien vaut $u_0$ ?
		
		\item La somme des 152 premiers éléments d'une suite arithmétique de raison $r = 4$ vaut 1312. Quel est le premier élément de cette suite ?
		
		\item Vérifie que $a$, $2a - b$ et $3a - 2b$ sont trois termes consécutifs d'une progression arithmétique.
		
		\item Calcule la somme des 100 premiers nombres dont l'écriture se termine par 3 ou 8.
		
		\item Soit la suite géométrique $u$ définie par $u_0 = 3$ et $q = -2$. Calcule $u_{224}$.
		
		\item Soit $u$ une suite géométrique. Comment calculer $u_{24}$ en connaissant $u_{23}$ et $u_{25}$ ?
		
		\item Les nombres $-5, a, -45$ sont trois termes consécutifs d'une progression géométrique. Calcule $a$.
		
		\item Un pendule a été lâché à 20 cm de sa position d'équilibre. On exprime l'amplitude des mouvements par des nombres positifs à droite et des nombres négatifs à gauche de cette position d'équilibre. A cause des forces de frottement, l'amplitude de chaque mouvement vaut, en valeur absolue, 80\% de celle du mouvement précédent.
			\begin{enumerate}
				\item Ecris les amplitudes $a_0, \dots, a_4$ des 5 premiers mouvements.
				\item Donner la formule de récurrence qui exprime $a_{n+1}$ en fonctione de $a_n$.
				\item Déduis la valeur de $a_n$ en fonction de $a_0$.
			\end{enumerate}
	
	\end{enumerate}
	
\end{document}