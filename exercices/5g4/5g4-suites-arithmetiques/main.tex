\documentclass[12pt]{article}

\usepackage{fontspec}
\usepackage{amsmath, amssymb, amsfonts}
\usepackage{graphicx}
\usepackage{hyperref}
\usepackage{color}
\usepackage{geometry}
\geometry{a4paper}
\usepackage{pgfplots}

\begin{document}

	\vspace{3em}
	\begin{center}
			{\Large Exercices : les suites arithmétiques}
	\end{center}
	
	\vspace{3em}
	
	\textbf{Rappel des formules :}
	\begin{itemize}
		\item $u_n = u_{n-1} + r$
		\item $u_n = u_0 + n \cdot r$
		\item $S_N = \frac{N \cdot (u_0 + u_{N-1})}{2}$
	\end{itemize}
	
	\vspace{3em}
	
	\begin{enumerate}
		\item 
			Soit la suite arithmétique $u$ définie par $u_0 = 3$ et $r=4$. Calcule $u_1$, $u_2$ et $u_3$.
			\vspace{3em}
			
		\item 
			Soit la suite arithmétique $u$ définie par $u_0 = -3$ et $r=3$. Calcule $u_{253}$.
			\vspace{3em}
			
		\item 
			Soit la suite arithmétique $u$. On sait que $u_{13} = 21$ et que $u_{14} = 23$. Calcule $u_{10}$.
			\vspace{3em}
			
		\item 
			Est-ce que les suites suivantes sont arithmétiques ? Si oui, donne la raison.
			\begin{enumerate}
				\item $u_0 = -3, \quad u_1 = -1, \quad u_2 = 1, \quad u_3 = 3, \quad u_4 = 5,\dots$
				\item $u_0 = 1, \quad u_1 = 2, \quad u_2 = 4, \quad u_3 = 8, \quad u_4 = 16,\dots$
			\end{enumerate}
			\vspace{3em}
			
		\item 
			Soit une suite $u$. On a $u_0 = 1$, $u_1=2$, $u_2 = 5$ et $u_3 = 14$. Calcule $\sum_{n=0}^{3} u_n$.
			\vspace{3em}
			
		\item 
			Soit la suite arithmétique $u$ définie par $u_0 = 17$ et $r = 5$. Calcule $S_{117}$, la somme les 117 premiers éléments de la suite $u$.
	\end{enumerate}
	
\end{document}