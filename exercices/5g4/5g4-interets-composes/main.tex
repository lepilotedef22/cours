\documentclass[12pt]{article}

\usepackage{fontspec}
\usepackage{amsmath, amssymb, amsfonts}
\usepackage{graphicx}
\usepackage{hyperref}
\usepackage{color}
\usepackage{geometry}
\geometry{a4paper}

\usepackage{tikz}
\usetikzlibrary{calc, positioning}

\begin{document}

	\vspace{3em}
	\begin{center}
			{\Large Exercices : les intérêts composés}
	\end{center}
	
	\vspace{3em}
	
	\textbf{Rappel des formules :}
	\begin{itemize}
		\item $C_n = C_0 \cdot \left(1 + i\right)^n$
	\end{itemize}

	\vspace{3em}
	
	\begin{enumerate}
		\item 
			On dépose 250€ sur un compte d'épargne avec 2,5\% de taux d'intérêt par an.
			\begin{enumerate}
				\item Quel sera le capital après 5 ans ?
				\item Quel sera le capital après 12 ans ?
			\end{enumerate}
			\vspace{3em}
			
		\item 
			On dépose 1500€ sur un compte d'épargne avec 4\% de taux d'intérêt par an.
			\begin{enumerate}
				\item Quel est le bénéfice réalisé entre la quatrième et la cinquième année ?
				\item Quel est le bénéfice total après 10 ans ?
			\end{enumerate}
			\vspace{3em}
			
		\item 
			On veut acheter un vélo à 250€. On a 180€ qu'on dépose sur un compte d'épargne avec 15\% de taux d'intérêt par an. Combien de temps faut-il attendre pour pouvoir acheter le vélo ?
			
	\end{enumerate}
	
\end{document}