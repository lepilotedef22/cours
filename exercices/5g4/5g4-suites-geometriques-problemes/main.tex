\documentclass[12pt]{article}

\usepackage{fontspec}
\usepackage{amsmath, amssymb, amsfonts}
\usepackage{graphicx}
\usepackage{hyperref}
\usepackage{color}
\usepackage{geometry}
\geometry{a4paper}

\usepackage{tikz}
\usetikzlibrary{calc, positioning}

\begin{document}

	\vspace{3em}
	\begin{center}
			{\Large Applications : les suites géométriques}
	\end{center}
	
	\vspace{3em}
	
	\begin{enumerate}
		\item 
			Une population de bactéries double tous les trois jours. Le premier jour il y a 4 bactéries.
			\begin{enumerate}
				\item Combien y aura-t-il de bactéries après 18 jours ?
				\item Combien de jours faudra-t-il attendre pour que la population dépasse les 5000 bactéries ?
			\end{enumerate}
			\vspace{3em}
			
		\item 
			La radio-activité d'un matériau est divisée par deux après une durée appelée \textit{temps de demi-vie}. Le temps de demi-vie de l'uranium présent dans les centrales nucléaires en Belgique est de \textbf{703 millions d'années}. Combien de temps faut-il attendre pour que sa radio-activité soit \textbf{divisée par 8} ?
			
		
	\end{enumerate}
	
\end{document}