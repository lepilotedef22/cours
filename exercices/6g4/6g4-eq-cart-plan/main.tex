\documentclass[12pt]{article}

\usepackage{fontspec}
\usepackage{amsmath, amssymb, amsfonts}
\usepackage{graphicx}
\usepackage{hyperref}
\usepackage{color}
\usepackage{geometry}
\geometry{a4paper}

\usepackage{tikz}
\usetikzlibrary{calc, positioning}

\title{Exercices : équations cartésiennes de plans}
\date{}

\begin{document}

	\maketitle
	
	\section{Rappel de la formule}
		\begin{itemize}
			\item Soit un vecteur $\vec{n} = \left(a;b;c\right)$ normal au plan $P$.
			\item $P \equiv ax + by + cz + d = 0$.
		\end{itemize}

	\section{Exercices}
		\begin{enumerate}
			\item Soit le plan $P$ de vecteur normal $\vec{n} = \left(1; 1; 1\right)$ et passant par le point $A \equiv \left(0; 0; 1\right)$. Donne l'équation cartésienne du plan $P$.
			
			\item Soit les trois points $A \equiv \left(1; 0; 2\right)$, $B \equiv \left(0; 0; 1\right)$ et $C \equiv \left(0; 2; 1\right)$. Donne l'équation cartésienne d'un plan qui passe par $A$, $B$ et $C$.
			
			\item Est-ce que les plans suivants sont sécants ? Orthogonaux ? Parallèles ? Confondus ? Justifie.
			\begin{enumerate}
				\item $P \equiv 2x + 3y -2z = 0 $ et $P' \equiv 3x - 2y + 2z = 0$
				\item $P \equiv x - 2 = 0 $ et $P' \equiv z - 3 = 0$
				\item $P \equiv x + y + z -3 = 0 $ et $P' \equiv 3x + 3y + 3z - 1 = 0$
				\item $P \equiv -x + 2y - 3 = 0 $ et $P' \equiv x - 2y + 3 = 0$
			\end{enumerate}
		\end{enumerate}
	
\end{document}^