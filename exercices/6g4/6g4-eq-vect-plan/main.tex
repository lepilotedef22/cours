\documentclass[12pt]{article}

\usepackage{fontspec}
\usepackage{amsmath, amssymb, amsfonts}
\usepackage{graphicx}
\usepackage{hyperref}
\usepackage{color}
\usepackage{geometry}
\geometry{a4paper}

\usepackage{tikz}
\usetikzlibrary{calc, positioning}

\begin{document}

	\vspace{3em}
	\begin{center}
			{\Large Exercices : équations vectorielles de plans}
	\end{center}
	
	\vspace{3em}
	
	\textbf{Rappel de la formule :}
	\begin{itemize}
		\item $P \equiv k_1 \cdot \vec{u} + k_2 \cdot \vec{v} + \vec{OP},~ (k_1, k_2 \in \mathbb{R})$
	\end{itemize}

	
	\begin{enumerate}
		\item Soit le plan $P \equiv k_1 \cdot \left(1; 2; -1\right) + k_2 \cdot \left(-1; -1; -3\right) + \left(0; 0; 1\right), ~ (k_1, k_2 \in \mathbb{R})$. Trouve l'équation du plan $R$ qui passe par le point $\left(0; 0; 1\right)$, dont un vecteur directeur est $\left(1; 2; -1\right)$ et qui est orthogonal à $P$.
			
	\end{enumerate}
	
\end{document}